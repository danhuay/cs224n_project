\documentclass{article}

\usepackage[final]{neurips_2019}

\usepackage[utf8]{inputenc}
\usepackage[T1]{fontenc}
\usepackage{hyperref}
\usepackage{url}
\usepackage{booktabs}
\usepackage{amsfonts}
\usepackage{nicefrac}
\usepackage{microtype}
\usepackage{graphicx}
\usepackage{xcolor}
\usepackage{lipsum}

\newcommand{\note}[1]{\textcolor{blue}{{#1}}}

\title{
  Title of your project \\
  \vspace{1em}
  \small{\normalfont Stanford CS224N \{Custom, Default\} Project}  % Select one and delete the other
}

\author{
  Name \\
  Department of Computer Science \\
  Stanford University \\
  \texttt{name@stanford.edu} \\
  % Examples of more authors
%   \And
%   Name \\
%   Department of Computer Science \\
%   Stanford University \\
%   \texttt{name@stanford.edu} \\
%   \And
%   Name \\
%   Department of Computer Science \\
%   Stanford University \\
%   \texttt{name@stanford.edu}
}

\begin{document}

\maketitle

\begin{abstract}
  Your abstract should motivate the problem, describe your goals, and highlight your main findings. Given that your project is still in progress, it is okay if your findings are what you are still working on.
\end{abstract}


\section{Key Information to include}
\begin{itemize}
    \item TA mentor:
    \item External collaborators (if no, indicate ``No''):
    \item External mentor (if no, indicate ``No''):
    \item Sharing project (if no, indicate ``No''):
\end{itemize}

% {\color{red} This template does not contain the full instruction set for this assignment; please refer back to the milestone instructions PDF.}

\section{Approach}
This section details your approach to the problem. 
\begin{itemize}
    \item Please be specific when describing your main approaches. You may want to include key equations and figures (though it is fine if you want to defer creating time-consuming figures until the final report).
    \item Describe your baselines. Depending on space constraints and how standard your baseline is, you might do this in detail or simply refer to other papers for details. Default project teams can do the latter when describing the provided baseline model.
    \item If any part of your approach is original, make it clear. For models and techniques that are not yours, provide references.
    \item If you are using any code that you did not write yourself, make it clear and provide a reference or link. 
    When describing something you coded yourself, make it clear.
\end{itemize} 


\section{Experiments}
This section is expected to contain the following.
\begin{itemize}
    \item \textbf{Data}: Describe the dataset(s) you are using along with references. Make sure the task associated with the dataset is clearly described.
    \item \textbf{Evaluation method}: Describe the evaluation metric(s) you used, plus any other details necessary to understand your evaluation.
    \item \textbf{Experimental details}: Please explain how you ran your experiments (e.g. model configurations, learning rate, training time, etc.).
    \item \textbf{Results}: Report the quantitative results that you have so far. Use a table or plot to compare multiple results and compare against your baselines.
\end{itemize}


\section{Future work}
Describe what you plan to do for the rest of the project and why.


\bibliographystyle{unsrt}
\bibliography{references}

\end{document}
